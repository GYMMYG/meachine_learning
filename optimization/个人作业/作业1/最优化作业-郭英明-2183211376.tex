\documentclass[UTF8]{ctexart}

\usepackage{amsmath}
\usepackage{mathtools}
\usepackage{fancyhdr} 
\pagestyle{fancy}
\lhead{}
\chead{}
\rhead{}
\lfoot{}
\cfoot{}
\rfoot{\thepage}
\renewcommand{\headrulewidth}{0pt} 
\renewcommand{\footrulewidth}{0pt} 

\title{Theory and method of optimization}
\author{郭英明}
\date{2020/10/1}

\begin{document}

\maketitle
\tableofcontents

\newpage

\section{The first homework}
\subsection{}
Question: 
the intersection of convex sets is convex.

Answer:

假设凸集为 $S_1,S_2 \cdots S_{n}$,若证明 $S_1\cap S_2$ 为凸集,则 $S_1\cap S_2 \cdots \cap S_{n}$为凸集。

下证$S_1\cap S_2$ 为凸集:

设$x_1,x_2\in S_1\cap S_2$, $x_0 = \theta * x_1 + (1 - \theta ) * x_2, (0 \leq \theta \leq 1)$,

因为$x_1 \in S_1$,$x_2 \in S_1$,且$S_1$为凸集,

所以$x_0 = \theta * x_1 + (1 - \theta ) * x_2 \in S_1$,

同理,$x_0 \in S_2$,

所以,$,x_0\in S_1\cap S_2$。

所以,对$\forall x_1,x_2 \in S_1 \cap S_2,x_0 = \theta * x_1 + (1 - \theta ) * x_2 \in S_1 \cap S_2, (0 \leq \theta \leq 1)$。

所以$S_1\cap S_2$ 为凸集,

得证。

\subsection{}
Question: 
if $C$ is convex,then
\begin{center}
$aC + b = {ax + b:x \in C}$
\end{center}
is convex for any $a$,$b$

Answer:

对$\forall x_1,x_2 \in C$,

$ax_1 + b \in aC + b,ax_2 + b \in aC + b$

设$x_0 = \theta * x_1 + (1 - \theta ) * x_2, (0 \leq \theta \leq 1)$,

则\begin{align}\nonumber
ax_0 + b &= a*(\theta * x_1 + (1 - \theta ) * x_2) + b \\
\nonumber &= \theta * (ax_1 + b) + (1 - \theta) * (ax_2 + b)
\end{align}

所以对$\forall ax_1 + b \in aC + b,ax_2 + b \in aC + b$,

$ax_0 + b = \theta * (ax_1 + b) + (1 - \theta) * (ax_2 + b) \in aC + b$

所以$aC + b$为凸集。
\subsection{}
Question:
if $f(x) = Ax + b$ and $C$ is convex,
and if $D$ is convex then
\begin{center}
$f^{-1}(D) = {x:f(x) \in D}$
\end{center}
is convex

Answer:

因为$D$为凸集,$b$为常数,

所以,$D - b$为凸集。

对$\forall D_1,D_2 \in D - b$,

设
\begin{align}\nonumber
Ax_1 &= D_1\\
\nonumber Ax_2 &= D_2\\
\nonumber x_0 &= \theta * x_1 + (1 - \theta ) * x_2), (0 \leq \theta \leq 1)
\end{align}

所以$Ax_0 = \theta * D_1 + (1 - \theta) * D_2 \in D - b$

所以$x_0 \in$集合$f^{-1}(D)$

所以集合$f^{-1}(D)$为凸集。

\subsection{}
Question:
log-sum-exp:$f(x) = \log{\sum_{i=1}^n \exp{x_k}}$ is convex

Answer:
\begin{center}
$\nabla f(x) = \begin{bmatrix}
\frac{\exp x_1}{\sum_{i=1}^n \exp x_i}&\frac{\exp x_2}{\sum_{i=1}^n \exp x_i}&\cdots&\frac{\exp x_n}{\sum_{i=1}^n \exp x_i}
\end{bmatrix} * x$
\end{center}

二阶导分母相同且为正,省去后:$\nabla^2 f(x) =$
\begin{center}$ \begin{bmatrix} \begin{smallmatrix}
\exp x_1\sum_{i=1}^n \exp x_i - \exp x_1\exp x_1  &  -\exp x_1\exp x_2 & \cdots & -\exp x_1\exp x_n \\
 -\exp x_1\exp x_2 & \exp x_2\sum_{i=1}^n \exp x_i - \exp x_2\exp x_2 & \cdots & -\exp x_2\exp x_n \\
 \vdots   & \vdots & \ddots  & \vdots  \\
-\exp x_1\exp x_n &   \cdots &\cdots  &\exp x_n\sum_{i=1}^n \exp x_i - \exp x_n\exp x_n    
\end{smallmatrix} \end{bmatrix}$
\end{center}

$n = 1$时矩阵为半正定,

假设$n = k - 1$时矩阵为半正定阵,记为矩阵$A$,

则$n = k$时矩阵的$k - 1$阶主子式可写为:

\begin{center}
$A + \begin{bmatrix} 
\exp x_1exp x_k  & 0 & \cdots & 0 \\
0 & \exp x_2\exp x_k & \cdots & 0 \\
 \vdots   & \vdots & \ddots  & \vdots  \\
0 &   \cdots &\cdots  &\exp x_{k-1}\exp x_k   
 \end{bmatrix}$
\end{center}

因为半正定具有可加性,

所以$n = k$时的$k - 1$阶顺序主子式为半正定矩阵,

所以$n = k$时的前$k - 1$阶顺序主子式均大于等于$0$,

对$\nabla^2 f(x)$做初等行变换,

\begin{center}
$\begin{bmatrix} \begin{smallmatrix}
0  &  0 & \cdots &0 \\
 -\exp x_1\exp x_2 & \exp x_2\sum_{i=1}^n \exp x_i - \exp x_2\exp x_2 & \cdots & -\exp x_2\exp x_n \\
 \vdots   & \vdots & \ddots  & \vdots  \\
-\exp x_1\exp x_n &   \cdots &\cdots  &\exp x_n\sum_{i=1}^n \exp x_i - \exp x_n\exp x_n    
\end{smallmatrix} \end{bmatrix}$
\end{center}

按第一行展开,行列式值为$0$,

所以$n = k$时的前$k$阶顺序主子式均大于等于$0$,

所以$\nabla^2 f(x)$半正定。

所以$f(x)$为凸函数。

\section{The second homework}
\subsection{}
Question:
证明无穷范数$|\lvert X |\lvert_\infty := max|x_i|$满足范数的三个性质。

Answer:

性质一:$|\lvert X |\lvert_\infty := \max|x_i| \ge 0$,当且仅当$X = 0$时等号成立。

性质二:
\begin{align}\nonumber
|\lvert tX |\lvert_\infty &= \sqrt[n]{(tx_1)^n + (tx_2)^n + \cdots +(tx_n)^n} \\  \nonumber
                          &= t * \sqrt[n]{(x_1)^n + (x_2)^n + \cdots +(x_n)^n} \\  \nonumber         
                          &= t\max \limits_{0\leq i \leq n}|x_i|               \\   \nonumber
                          &= t|\lvert X |\lvert_\infty
\end{align}

性质三:
\begin{align}\nonumber
|\lvert X + Y |\lvert_\infty &= \sqrt[n]{(x_1 + y_1)^n + (x_2 + y_2)^n + \cdots +(x_n + y_n)^n} \\  \nonumber
                             &= \max \limits_{0\leq i \leq n}|x_i + y_i|                        \\  \nonumber
                       &\leq \max \limits_{0\leq i \leq n}|x_i| + \max \limits_{0\leq i \leq n}|y_i| \\ \nonumber
               &\leq \max \limits_{0\leq i \leq n}|x_i| + \max \limits_{0\leq j \leq n}|y_j| \\ \nonumber
                       &= |\lvert X |\lvert_\infty + |\lvert Y |\lvert_\infty
\end{align}
\subsection{}
Question:
凸函数的局部最优为全局最优

Answer:

设$x_0$为全局最优点,若存在局部最优点$x_1$,

若$f(x_0) = f(x_1)$,

则局部最优为全局最优。

若$f(x_0) < f(x_1)$,
不妨设$x_0 < x_1$。

因为$x_1$为局部最优,

所以存在$x_2$使$x_0 < x_2 <x_1$,$f(x_2) > f(x_1) > f(x_0)$,

设$x_2 = \theta x_0 +  \theta x_1,(0 \leq \theta \leq 1)$,

$f(x_2) > \theta f(x_0) + (1 - \theta)(f(x_1)$

与$f(x)$为凸函数矛盾。

所以$f(x_0) = f(x_1)$,$x_1$为全局最优。

\section{The third homework}
\subsection{}
Question:
无穷范数的对偶范数是一范数

Answer:
\begin{align}\nonumber
|\lvert u |\lvert_* &= \sup \limits_{|\lvert v |\lvert_\infty \leq 1} \mathbf{u}^\mathrm{T}v  \\ \nonumber
                    &= \frac{\sum_{i=1}^n u_iv_i}{\max \limits_{0\leq j \leq n}|v_j|}  \\ \nonumber
               & \leq \frac{\sum_{i=1}^n |u_i||v_i|}{\max \limits_{0\leq j \leq n}|v_j|}  \\ \nonumber
     & \leq \frac{\sum_{i=1}^n |u_i|\max \limits_{0\leq j \leq n}|v_j|}{\max \limits_{0\leq j \leq n}|v_j|}  \\ \nonumber  &= \sum_{i=1}^n |u_i|  \\ \nonumber
          &= |\lvert u |\lvert_1
\end{align}
所以,无穷范数的对偶范数是一范数。

\section{The fourth homework}
\subsection{}
Question:
求一范数的次梯度

Answer:

$f(x) = |\lvert X |\lvert_1 = \sum_{i = 1}^n |x_i|$

当$x_i > 0$时,

$ \frac{\partial f}{\partial x_i}=\frac{\partial x_i}{\partial x_i}= 1  $

当$x_i < 0$时,

$ \frac{\partial f}{\partial x_i}=\frac{-\partial x_i}{\partial x_i}= -1  $

当$x_i = 0$时,

$\partial {f(x)} = \left\{g|\mathbf{g}^\mathrm{T}(y - x) \leq f(y) - f(x)\right\}$

当$y_i > x_i$时,其它项相等,

$\mathbf{g}^\mathrm{T}_i \leq \frac{f(y) - f(x)}{y_i - x_i} = \frac{y_i - 0}{y_i - 0} = 1$

同理,当$y_i < x_i$时,

$\mathbf{g}^\mathrm{T}_i \ge \frac{f(y) - f(x)}{y_i - x_i} = \frac{-y_i - 0}{y_i - 0} = -1$

所以$-1 \leq \frac{\partial f}{\partial x_i} \leq 1$

综上:

$$ \frac{\partial f}{\partial x_i} = \left\{
\begin{aligned}
&1  &\text{if} \quad  x_i > 0  \\
&-1 &\text{if} \quad  x_i < 0  \\
&\left[-1,1 \right] &\text{if} \quad  x_i = 0
\end{aligned}
\right.
$$
\end{document}