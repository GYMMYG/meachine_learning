\documentclass[UTF8]{ctexart}

\usepackage{amsthm}
\usepackage{amsmath}
\usepackage{amssymb}
\usepackage{mathtools}
\usepackage{fancyhdr} 
\pagestyle{fancy}
\lhead{}
\chead{}
\rhead{}
\lfoot{}
\cfoot{}
\rfoot{\thepage}
\renewcommand{\headrulewidth}{0pt} 
\renewcommand{\footrulewidth}{0pt}

\title{机器学习作业}
\author{郭英明}
\date{2020/10/26}



\begin{document}
\maketitle
\section{}
写出下列线性规划问题的对偶问题
\begin{center}
$\min \limits_{x\in\mathbb{R}^2} c^{T}x$
\begin{align}
s.t.& \bar{A}x-\bar{b}\leq0,\\
&Ax-b=0
\end{align}
其中,$c\in\mathbb{R}^2,\bar{A}\in\mathbb{R}^{m*n},\bar{b}\in\mathbb{R}^{m},{A}\in\mathbb{R}^{p*n},{b}\in\mathbb{R}^{p}$
\end{center}

\noindent\textbf{解答:}
对应的拉格朗日函数为:
\begin{center}
$L(x,\lambda,\upsilon)=c^{T}x+\lambda^{T}(\bar{A}x-\bar{b})+\upsilon^{T}(Ax-b)$\\
\end{center}
拉格朗日函数的最小值函数为:
\begin{align}
g(\lambda,\upsilon)&=\inf_{x}L(x,\lambda,\upsilon)\\
&=\inf_{x}(c+\bar{A}^{T}\lambda+A^{T}\upsilon)^{T}x-\bar{b}^{T}\lambda-b^{T}\upsilon
\end{align}

所以可以得到:\\
\begin{equation}
g(\lambda,\upsilon)=\left\{
\begin{aligned}
&-\bar{b}^{T}\lambda-b^{T}\upsilon,c+\bar{A}^{T}\lambda+A^{T}\upsilon=0\\
&-\infty,others\\
\end{aligned}
\right.
\end{equation}

所以对偶问题可以写成:
\begin{center}
$\max \limits_{\lambda,\upsilon}-\bar{b}^{T}\lambda-b^{T}\upsilon$
\begin{align}
\boldsymbol{s.t.}  c+\bar{A}^{T}\lambda+A^{T}\upsilon&=0,\\
\lambda&\ge0
\end{align}
\end{center}


\section{}
软间隔支持向量机还可以定义为一下形式:
\begin{center}
$\min \limits_{\omega,b,\xi} \frac{1}{2}\left \| \boldsymbol{\omega} \right \|^2+C\sum_{i=1}^m \xi_{i}^2$\\
$\boldsymbol{s.t.} y_i (\boldsymbol{\omega}^{T}\boldsymbol{x}_i+b)\ge1-\xi_i,i=1,\cdots,m.$
\end{center}
试求其对偶形式,写出其$KKT$条件,并根据$KKT$条件讨论训练样本的分布情况\\
\textbf{解答:}\\
对应的拉格朗日函数为:
\begin{center}
$L(\omega,b,\xi,\lambda)= \frac{1}{2}\left \| \boldsymbol{\omega} \right \|^2+C\sum_{i=1}^m \xi_{i}^2+\sum_{i=1}^m \lambda_i(1-\xi_i-y_i(\boldsymbol{\omega}^{T}\boldsymbol{x}_i+b)$\\
\end{center}
对于上述拉格朗日函数,若要求得其最小值函数,对于$\boldsymbol\omega,b,\xi_i$分别求偏导数并使之等于0得到如下式子:

\begin{center}

$\frac{\partial L}{\partial\boldsymbol\omega}=\boldsymbol\omega-\sum_{i=1}^m \lambda_{i}y_{i}\boldsymbol{x}_i=0$\\
$\frac{\partial L}{\partial b}=\sum_{i=1}^m \lambda_{i}y_{i}=0$\\
$\frac{\partial L}{\partial \xi_i}=2C\xi_i-\lambda_i=0$
\end{center}
将上述式子代入拉格朗日函数即可得对偶问题为:
\begin{center}
$\max \limits_{\lambda}\sum_{i=1}^m \lambda_i-\frac{1}{2}\sum_{i=1}^m \sum_{j=1}^m\lambda_{i}\lambda_{j}y{i}y{j}\boldsymbol{x}^{T}\boldsymbol{x}-\frac{1}{4C}\sum_{i=1}^m \lambda_i^2$
\begin{align}
\boldsymbol{s.t.}  \sum_{i=1}^m \lambda_{i}y{i} & =0,\\
\lambda & \ge0
\end{align}
\end{center}
其$KKT$条件如下:


\begin{equation}
\left\{
\begin{aligned}
&\lambda_i  \ge  0 \\
&1-\xi_i-y_i(\boldsymbol{\omega}^{T}\boldsymbol{x}_i+b \leq  0 \\
&\lambda_i(1-\xi_i-y_i(\boldsymbol{\omega}^{T}\boldsymbol{x}_i+b)  =  0
\end{aligned}
\right.
\end{equation}
训练样本的分布情况如下:\\
{$\lambda_i=0$时}
\noindent$y_i(\boldsymbol{\omega}^{T}\boldsymbol{x}_i+b>1-\xi_i$,则该样本不会对$f(x)$有任何影响

{$\lambda_i>0$时}
\noindent$y_i(\boldsymbol{\omega}^{T}\boldsymbol{x}_i+b=1-\xi_i$,则该样本是支持向量。\\
当$0<\xi<1$即$0<\frac{\lambda_i}{2C}<1$时,则该样本落在最大间隔内部\\
当$\xi>1$即$\frac{\lambda_i}{2C}>1$时,则该样本被错误分类









\end{document}